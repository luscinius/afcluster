\documentclass[12pt]{article}

\setlength{\topmargin}{0.0cm}
\setlength{\textheight}{21.5cm}
\setlength{\oddsidemargin}{0cm}
\setlength{\textwidth}{16.5cm}
\setlength{\columnsep}{0.6cm}

\newcommand{\Dc}{\mathcal{D}}
\newcommand{\Oc}{\mathcal{O}}
\newcommand{\Lc}{\mathcal{L}}

\newcommand{\DKL}{D_{\mathrm{KL}}}

\pagestyle{empty}

\begin{document}

\section*{afcluster software}
\emph{afcluster} software performs centroid based clustering
of nucleotide sequences based on word ($n$-mer) counts.
Word counts can be computed using overlapping or non-overlapping $n$-mers,
optionally concatenating the sequence together with its reverse complement.

Implemented clustering algorithms include
$k$-means type algorithm and expectation maximization (EM) algorithm.
For $k$-means type algorithm one can use the following distances:
$L_2$ (Euclidean) distance, Kullback-Leibler (KL) divergence, $d_2$ distance,
$d_2^*$ distance, $\chi^2$ statistic and the symmetrized KL divergence. 

One can also perform consensus clustering.
Regular clustering is performed a specified number of times,
and the consensus partitioning is built based on patterns
of individual samples clustering together.
Consensus clustering mitigates the dependence of the resulting partitioning
on the random initialization inherent to centroid-based methods.
This is achieved at the cost of $\Oc(N^2 \log N)$ time complexity and $\Oc(N^2)$
space complexity for input consisting of $N$ sequences.

The software also allows soft EM clustering,
in which case each sequence is only assigned to each cluster with some probability.
This method gives some estimate of the clustering accuracy
without the overhead of the consensus clustering.

\emph{afcluster} software is implemented in C++.
It has been compiled and tested using GNU GCC.
The tool is open source under the GNU GPL license.

\end{document}
